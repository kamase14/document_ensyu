\documentclass{jsarticle}

\begin{document}
 
 \begin{titlepage}
  \title{難病患者の"じりつ"を育む支援の構築に向けて 〜ALS患者を通して〜のまとめ}
  \author{12-423 田中 寿明}
 \end{titlepage}
 
 \section{ALSとは}
 ALSはAmyotrophic Leteral Sclerosisの略である、文中のAmyotrophicとは筋萎縮性、Lateral Sclerosisは脊髄の側索が硬くなる病気を意味する。\\
 文字通り筋肉が萎縮していく難病であるが、筋肉そのものが悪いわけではなく、筋肉を支配する神経が悪くなった結果として筋肉が萎縮していく、という神経の病気である。
 
 \section{ケアの方法}
 この病気は進行性であり、だんだんと神経が変性していき死滅してしまうと、筋肉がやせ細って機能しなくなってしまう。\\
 そうすると適切な処置が難しくなるので、適切な時期に適切な処置を行うことが肝要となる。\\
 
 例えるならば、口の筋肉が弱くなり食事ができなくなるならば、胃に小さな穴を開け、そこから栄養を注入することで栄養失調を防ぐ。呼吸障害については人工呼吸器を用いて呼吸を助ける、といった具合である。
 
 \section{東日本大震災を通して}
 東日本大震災発生当時、今井氏は国立病院機構宮城病院に所属していた。\\
 その時、病院の中には317名の入院患者が取り残されており、大半のライフラインが死んでいる中で患者への対応をせねばならなかった。\\
 また、津波の再来のおそれから一階の患者を二階に移す際には廊下や食堂までベッドが並び、戦場の前線のようだったという。\\
 
 人工呼吸器をつけた患者さんはケアが多く、余震が続く中でケアが持続できる保証もなかったので、自衛隊に依頼し、状態の安定している患者さんを東京や山形・新潟の病院に移すこととなった。\\
 
 しばらくして、東京大学に搬送した患者から連絡が来た。気管を切開しており、コミュニケーションが非常に難しい患者さんであるが、意思伝達装置というコンピュータを使って書いたそうだった。\\
 70歳と高齢な患者さんでありながら、コンピュータを操る姿は東京大学ではびっくりされたようです。
 
 \section{ALS患者とのコミュニケーション}
 前述の通り、ALSはだんだんと手足が動かなくなり、言葉も話せなくなる病気である。\\
 そこで、どのようにしてコミュニケーションを取るか、という問題が発生する。\\
 
 意思伝達装置というコンピュータを使えばコミュニケーションが出来るのだが、高齢の人では使えない人が多い。そうなってしまうと、だんだんとコミュニケーションができなくなってしまう。\\
 
 口が動けばある程度口パクで意思を伝えることができるが、誰にでも読み取れるわけではない。
 そうなってしまうと介護者が限定され、特定の介護者に負担が集中した結果、在宅介護が破綻してしまうことがある。
 
 負担軽減の方法として、
 \begin{itemize}
  \item 患者本人が主体となってケア内容を調整する。
  \item 往診医を確保する。
  \item ナースコールを押す前にコンピュータで内容を書いてもらう
 \end{itemize}
 というものが挙げられる。
 

\end{document}


